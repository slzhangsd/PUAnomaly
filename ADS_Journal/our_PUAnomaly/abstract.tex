%!TEX root = ../cluster_semi.tex
\begin{abstract}
To ensure the reliability of software applications and systems, Internet-based services monitor and detect anomalies for their KPI streams (Key Performance Indicator, such as CPU utilization, number of queries per second, response latency). %However,
Neither unsupervised nor supervised learning methods meet the requirements of anomaly detection because of the large number of KPI streams and the lack of labels. Semi-supervised learning methods ensure the accuracy of anomaly detection with reducing the manual labeling effort, however, all samples in specific time series  segments need to be carefully labeled.
In this paper, we propose \name{}, which integrates both PU learning and active learning techniques.
% PU learning enables us to train an anomaly detection model by only randomly labeling a few anomalies of several typical historical/existing KPI streams.
PU learning allows us to train an anomaly detection model only by randomly labeling a small number of anomalies in a few typical historical/existing KPI streams.
Active learning is applied to select the most reliable samples for manual labeling in the iterations of PU learning.
%Through experiments,we show that
\name{} achieves the best F-score over 0.83 on 80 new KPI streams, almost as the same as a state-of-the-art supervised approach, greatly outperforming a state-of-the-art unsupervised approach. Compared with the semi-supervised method, the bset F-score of \name{} is about 0.19 higher than the semi-supervised.

\end{abstract}